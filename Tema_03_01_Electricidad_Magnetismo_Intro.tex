\documentclass[14pt]{beamer}
\usepackage{./Estilos/BeamerUVM}
\usepackage{./Estilos/ColoresLatex}
\usetheme{Warsaw}
\usecolortheme{seahorse}
%\useoutertheme{default}
\setbeamercovered{invisible}
% or whatever (possibly just delete it)
\setbeamertemplate{section in toc}[sections numbered]
\setbeamertemplate{subsection in toc}[subsections numbered]
\setbeamertemplate{subsection in toc}{\leavevmode\leftskip=3.2em\rlap{\hskip-2em\inserttocsectionnumber.\inserttocsubsectionnumber}\inserttocsubsection\par}
\setbeamercolor{section in toc}{fg=blue}
\setbeamercolor{subsection in toc}{fg=blue}
\setbeamercolor{frametitle}{fg=blue}
\setbeamertemplate{caption}[numbered]

\setbeamertemplate{footline}
\beamertemplatenavigationsymbolsempty
\setbeamertemplate{headline}{}


\makeatletter
\setbeamercolor{section in foot}{bg=gray!30, fg=black!90!orange}
\setbeamercolor{subsection in foot}{bg=blue!30}
\setbeamercolor{date in foot}{bg=black}
\setbeamertemplate{footline}
{
  \leavevmode%
  \hbox{%
  \begin{beamercolorbox}[wd=.333333\paperwidth,ht=2.25ex,dp=1ex,center]{section in foot}%
    \usebeamerfont{section in foot} \insertsection
  \end{beamercolorbox}%
  \begin{beamercolorbox}[wd=.333333\paperwidth,ht=2.25ex,dp=1ex,center]{subsection in foot}%
    \usebeamerfont{subsection in foot}  \insertsubsection
  \end{beamercolorbox}%
  \begin{beamercolorbox}[wd=.333333\paperwidth,ht=2.25ex,dp=1ex,right]{date in head/foot}%
    \usebeamerfont{date in head/foot} {T1 - Segunda presentación} \hspace*{2em}
    \insertframenumber{} / \inserttotalframenumber \hspace*{2ex} 
  \end{beamercolorbox}}%
  \vskip0pt%
}
\makeatother

\makeatletter
\patchcmd{\beamer@sectionintoc}{\vskip1.5em}{\vskip0.8em}{}{}
\makeatother
% \usefonttheme{serif}
\usepackage[clock]{ifsym}
\DeclareSIUnit\erg{erg}
\DeclareSIUnit[number-unit-product = {\,}]\cal{cal}

\sisetup{per-mode=symbol}
\resetcounteronoverlays{saveenumi}

\title{\Large{Electricidad y Magnetismo} \\ \normalsize{Física 2}}
\date{17 de julio de 2023}

\begin{document}
\maketitle

\section*{Contenido}
\frame[allowframebreaks]{\frametitle{Contenido} \tableofcontents[currentsection, hideallsubsections]}

\section{Electricidad}
\frame{\tableofcontents[currentsection, hideothersubsections]}
\subsection{Electrostática}

\begin{frame}
\frametitle{Temas a revisar}
\setbeamercolor{item projected}{bg=bananayellow,fg=ao}
\setbeamertemplate{enumerate items}{%
\usebeamercolor[bg]{item projected}%
\raisebox{1.5pt}{\colorbox{bg}{\color{fg}\footnotesize\insertenumlabel}}%
}
\begin{enumerate}[<+->]
\item Carga eléctrica.
\item Conservación de carga.
\item Campo eléctrico.
\item Potencial eléctrico.
\seti
\end{enumerate}
\end{frame}
\begin{frame}
\frametitle{Temas a revisar}
\setbeamercolor{item projected}{bg=bananayellow,fg=ao}
\setbeamertemplate{enumerate items}{%
\usebeamercolor[bg]{item projected}%
\raisebox{1.5pt}{\colorbox{bg}{\color{fg}\footnotesize\insertenumlabel}}%
}
\begin{enumerate}[<+->]
\conti
\item Ley de Ohm.
\item Circuitos eléctricos.
\end{enumerate}
\end{frame}

\section{Magnetismo}
\frame{\tableofcontents[currentsection, hideothersubsections]}
\subsection{Temas a revisar}

\begin{frame}
\frametitle{Temas a revisar}
\setbeamercolor{item projected}{bg=burgundy,fg=white}
\setbeamertemplate{enumerate items}{%
\usebeamercolor[bg]{item projected}%
\raisebox{1.5pt}{\colorbox{bg}{\color{fg}\footnotesize\insertenumlabel}}%
}
\begin{enumerate}[<+->]
\item Magnetismo.
\item Campo magnético.
\item Relación entre electricidad y magnetismo.
\item El motor.
\end{enumerate}
\end{frame}

\section{Evaluación}
\frame{\tableofcontents[currentsection, hideothersubsections]}
\subsection{Elementos de evaluación}

\begin{frame}
\frametitle{Evaluación continua}
\setbeamercolor{item projected}{bg=cadetblue,fg=black}
\setbeamertemplate{enumerate items}{%
\usebeamercolor[bg]{item projected}%
\raisebox{1.5pt}{\colorbox{bg}{\color{fg}\footnotesize\insertenumlabel}}%
}
\begin{enumerate}[<+->]
\item Ejercicios de los temas.
\item Trabajo de desarrollo.
\item Ejercicios opcionales. 
\end{enumerate}
\end{frame}
\begin{frame}
\frametitle{Laboratorio}
Se trabajarán 3 prácticas con el tema de electricidad y magnetismo.
\\
\bigskip
\pause
Cada uno, con un correspondiente reporte de práctica.
\end{frame}
\begin{frame}
\frametitle{Trabajo de desarrollo}
Como en los bloques anteriores, desarrollarán un trabajo de desarrollo: \pause las aportaciones en el campo de la electricidad y magnetismo de:
\end{frame}
\begin{frame}
\frametitle{Trabajo de desarrollo}
\setbeamercolor{item projected}{bg=byzantine,fg=white}
\setbeamertemplate{enumerate items}{%
\usebeamercolor[bg]{item projected}%
\raisebox{1.5pt}{\colorbox{bg}{\color{fg}\footnotesize\insertenumlabel}}%
}
\begin{enumerate}[<+->]
\item Alessandro Volta.
\item Hans Cristian Oersted.
\item Michael Faraday.
\item André-Marie Ampere.
\item George Simon Ohm.
\item James Clerk Maxwell.
\end{enumerate}
\end{frame}
\begin{frame}
\frametitle{Puntaje a obtener}
La actividad de desarrollo, les otorgará hasta $10$ puntos de Evaluación Continua.
\end{frame}
\begin{frame}
\frametitle{Trabajo de desarrollo}
Se dejará un rúbrica para la evaluación del trabajo, y deberá de enviarse por Teams.
\\
\bigskip
\pause
La actividad se enviará a más tardar el 27 de julio a las 8 pm.
\end{frame}
\begin{frame}
\frametitle{El tercer examen}
El tercer y último examen parcial se presentará en la semana del 14 al 18 de agosto.
\end{frame}
\begin{frame}
\frametitle{Semanas de Trabajo}
Se tendrán 3 semanas de trabajo para cubrir la última parte del curso.
\\
\bigskip
\pause
Por lo que las sesiones serán como las que hemos tenido, de trabajo en clase y en casa.
\end{frame}
\begin{frame}
\frametitle{Trabajo Previo}
\textocolor{red}{No habrá oportunidad de reponer} actividades de Evaluación Continua y de Laboratorio.
\\
\bigskip
\pause
Por lo que se les invita a trabajar debidamente y entregar en tiempo las actividades.
\end{frame}
\begin{frame}
\frametitle{Seguimiento de actividades}
Se mantendrá un seguimiento de entrega de actividades para que no se comprometan en cuanto a la calificación del tercer parcial y de la asignatura.
\end{frame}
\end{document}