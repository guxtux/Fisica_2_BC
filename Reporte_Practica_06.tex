\documentclass[14pt]{extarticle}
\usepackage[utf8]{inputenc}
\usepackage[T1]{fontenc}
\usepackage[spanish,es-lcroman]{babel}
\usepackage{amsmath}
\usepackage{amsthm}
\usepackage{physics}
\usepackage{tikz}
\usepackage{float}
\usepackage[autostyle,spanish=mexican]{csquotes}
\usepackage[per-mode=symbol]{siunitx}
\usepackage{gensymb}
\usepackage{multicol}
\usepackage{enumitem}
\usepackage[left=2.00cm, right=2.00cm, top=2.00cm, 
     bottom=2.00cm]{geometry}
\usepackage{Estilos/ColoresLatex}
\usepackage{makecell}

\newcommand{\textocolor}[2]{\textbf{\textcolor{#1}{#2}}}
\DeclareSIUnit[number-unit-product = {\,}]\cal{cal}

%\renewcommand{\questionlabel}{\thequestion)}
\decimalpoint
\sisetup{bracket-numbers = false}

\title{\vspace*{-2cm} Práctica 6 Electrostática - Física 2\vspace{-5ex}}
\date{\today}

\begin{document}
\maketitle

\section{Para el reporte individual}

La elaboración y entrega del reporte en físico es INDIVIDUAL. Genera una portada en donde indicarás tu nombre completo.

Responde cada una de las preguntas del apartado \enquote{Investiga y escribe brevemente} de la Práctica 9 Electrostática del Manual.

Responde cada una de las siguientes preguntas.
\vspace*{0.5cm}
\begin{enumerate}
\item ¿Qué es un generador de Van de Graaff? ¿Cómo funciona?
\item ¿Por qué se levanta el pelo al tocar con la mano la superficie de la esfera del generador?
\item Cuando frotas un globo en el pelo y lo dejas en la pared, el globo se queda fijo. ¿Qué le sucede a la superficie de la pared? ¿Cómo recupera su equilibrio de cargas eléctricas?
\item Después de un rato de estar \enquote{pegado} el globo en la pared, el globo cae. ¿Por qué? ¿Dónde queda la carga eléctrica de la pared y el globo?
\item ¿Qué es un electroscopio?
\item Investiga en qué consisten las formas de electrizar un cuerpo por fricción o frotamiento, por contacto y por inducción y señala en qué momento de la actividad en clase se llevaron a cabo.
\end{enumerate}

Describe de manera detallada cada una de las actividades que realizaste, lo que encontraste y con las respuestas a las preguntas anteriores, redacta de manera clara un resumen de los resultados y evidencia encontrada.

Tu reporte se entregará en físico el día previo a la siguiente clase de Laboratorio, es decir, si tienes clase en martes, tu reporte se deberá de entregar el lunes previo a la clase, si tiene clase el jueves, entonces el miércoles previo a la clase, harás la entrega.

\end{document}