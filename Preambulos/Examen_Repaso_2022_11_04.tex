\documentclass[12pt, letter]{exam}
\usepackage[utf8]{inputenc}
\usepackage[T1]{fontenc}
\usepackage[spanish]{babel}
\usepackage{amsmath}
\usepackage{amsthm}
\usepackage{physics}
\usepackage{tikz}
\usepackage{float}
\usepackage{siunitx}
\usepackage{multicol}
\usepackage[left=2.00cm, right=2.00cm, top=2.00cm, 
     bottom=2.00cm]{geometry}

\begin{document}
\pagenumbering{arabic}
\textbf{Matemáticas.}
\begin{multicols}{2}
\begin{questions}
    \question ¿Cuál es el resultado de la siguiente operación?
    \\
    $\dfrac{2}{3} + \dfrac{1}{4} + \dfrac{2}{5} - \dfrac{3}{4} \divisionsymbol \dfrac{5}{2}$
    \begin{choices}
        \choice $- \dfrac{67}{50}$
        \choice $\dfrac{13}{15}$
        \choice $- \dfrac{23}{50}$
        \choice $\dfrac{7}{15}$
    \end{choices}
    \answerline
    \question ¿De qué número 15 es el $20\%$?
    \begin{choices}
        \choice $40$
        \choice $55$
        \choice $60$
        \choice $75$
    \end{choices}
    \answerline
    \question El resultado que se otiene al simplificar $\left( \dfrac{2}{3}\right)^{4}$ es:
    \begin{choices}
        \choice $\dfrac{16}{81}$
        \choice $\dfrac{8}{27}$
        \choice $\dfrac{16}{18}$
        \choice $\dfrac{8}{12}$
    \end{choices}
    \answerline
    \question La expresión $a + (a +1) + (a +2)$, en lenguaje común se lee como la suma de tres números
    \begin{choices}
        \choice impares.
        \choice pares.
        \choice fraccionarios.
        \choice consecutivos.
    \end{choices}
    \answerline
    \question María tiene cierta cantidad de dinero, Juan tiene el doble de María, y Pepe la mitad de María. Si juntos acumulan $\$3, 521.00$ ¿Cuánto dinero posee Juan?
    \begin{choices}
        \choice $\$2, 347.33$
        \choice $\$1, 1173.66$
        \choice $\$2, 012.00$
        \choice $\$1, 006.00$
    \end{choices}
    \answerline
    \question Dos veces un número más siete es igual a cinco veces ese número menos 17. ¿Cuál es ese número?
    \begin{choices}
        \choice $- \dfrac{1}{8}$
        \choice $- 8$
        \choice $- \dfrac{1}{8}$
        \choice $8$
    \end{choices}
    \answerline
    \question Compré $n$ lápices iguales, pagué con un billete de $\$50$ y recibí el cambio. Si planteo esta compra como $3.50 \, n + 4.50 = 50$, el número $3.50$ representa el:
    \begin{choices}
        \choice precio de un lápiz.
        \choice cambio que me dieron.
        \choice valor total de lápices.
        \choice número total de lápices.
    \end{choices}
    \answerline
    \question El valor e $x$ en la siguiente figura es:
    \begin{figure}[H]
        \centering
        \begin{tikzpicture}
            \draw [thick] (0, 0) -- (5, 0);
            \draw [thick] (0, -3) -- (5, -3);
            \draw [thick] (1.5, -4) -- (4, 1.5);
            \node at (2, 0.5) {$2 \, x + 4$};
            \node at (1.3, -3.5) {$\ang{70}$};
        \end{tikzpicture}
    \end{figure}
    \begin{choices}
        \choice $x = \ang{31}$
        \choice $x = \ang{33}$
        \choice $x = \ang{53}$
        \choice $x = \ang{55}$
    \end{choices}
    \answerline
    \question La bolsa de valores inició sus actividades de la semana con $28$ puntos; el lunes ganó $15$ puntos, el martes perdió $12$, el miércoles perdió $16$, el jueves ganó $40$ y el viernes perdió $18$. ¿Con cuántos puntos cerró la semana?
    \begin{choices}
        \choice $37$
        \choice $40$
        \choice $9$
        \choice $32$
    \end{choices}
    \answerline
    \question Un pueblo sufre una plaga de $10 000$ ratones que encuentran condiciones de reproducción al inicio de la temporada de cosecha, por lo que se espera un incremento de $120\%$ de su población. ¿Cuál es el total de ratones al final de la temporada?
    \begin{choices}
        \choice $12,000$
        \choice $22,000$
        \choice $120,000$
        \choice $220,000$
    \end{choices}
    \answerline
    \question En la ecuación $x - 3 = 6$, el valor de $x$ es:
    \begin{choices}
        \choice $x = -3$
        \choice $x = 2$
        \choice $x = 3$
        \choice $x = 9$
    \end{choices}
    \answerline
    \question Un terreno de forma rectangular tiene como medidas $\SI{16}{\meter}$ de largo y $\SI{12}{\meter}$ de ancho, respectivamente. La diagonal del terreno mide:
    \begin{choices}
        \choice $\SI{16}{\meter}$
        \choice $\SI{18}{\meter}$
        \choice $\SI{20}{\meter}$
        \choice $\SI{28}{\meter}$
    \end{choices}
    \answerline
\end{questions}
\end{multicols}
\end{document}