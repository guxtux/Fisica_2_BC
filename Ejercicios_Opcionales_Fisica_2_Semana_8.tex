\documentclass[14pt]{extarticle}
\usepackage[utf8]{inputenc}
\usepackage[T1]{fontenc}
\usepackage[spanish,es-lcroman]{babel}
\usepackage{amsmath}
\usepackage{amsthm}
\usepackage{physics}
\usepackage{tikz}
\usepackage{float}
\usepackage[autostyle,spanish=mexican]{csquotes}
\usepackage[per-mode=symbol]{siunitx}
\usepackage{gensymb}
\usepackage{multicol}
\usepackage{enumitem}
\usepackage[left=2.00cm, right=2.00cm, top=2.00cm, 
     bottom=2.00cm]{geometry}
\usepackage{Estilos/ColoresLatex}

\newcommand{\textocolor}[2]{\textbf{\textcolor{#1}{#2}}}
\DeclareSIUnit[number-unit-product = {\,}]\cal{cal}

%\renewcommand{\questionlabel}{\thequestion)}
\decimalpoint
\sisetup{bracket-numbers = false}

\title{\vspace*{-2cm} Ejercicios Opcionales - Física 2\vspace{-5ex}}
\date{\today}

\begin{document}
\maketitle

\section{Ejercicios a cuenta}

Con la finalidad de apoyar en la recuperación del promedio para los siguientes exámenes parciales, se dejarán una serie de \textocolor{red}{ejercicios adicionales} para Evaluación Continua.


Estos ejercicios serán de carácter \textocolor{cobalt}{opcional}, es decir, la alumna o alumno que desee resolverlos y enviarlos, les sumará $5$ puntos adicionales a la Evaluación Continua.

La entrega se hará vía Teams en asignación, teniendo como plazo el día domingo 16 de julio a las 8 pm.

Cada ejercicio vale $1$ punto, siempre y cuando esté correcto. Se otorgará una parte proporcional en caso de tener desarrollo detallado, pero el resultado no sea el esperado.

Anota en la hoja tu nombre completo, así como una identificación de cada ejercicio.

\begin{enumerate}

\item Un segmento de vía de ferrocarril de acero tiene una longitud de \SI{30000}{\meter} cuando la temperatura es de \SI{0.0}{\degreeCelsius} ¿Cuál es su longitud cuando la temperatura es de \SI{40.0}{\degreeCelsius}?
\item Un alambre telefónico de cobre en esencia no tiene comba (no se \enquote{cuelga}) entre postes separados \SI{35.0}{\meter} en un día de invierno cuando la temperatura es de \SI{-20.0}{\degreeCelsius}. ¿Cuánto más largo es el alambre en un día de verano, cuando la temperatura es de \SI{35.0}{\degreeCelsius}?
\item¿Cuál es la longitud de un riel de hierro a \SI{6}{\degreeCelsius} si a \SI{40}{\degreeCelsius} mide \SI{50}{\meter}? ¿Cuánto se contrajo?
\item Determina el calor específico de una muestra metálica de \SI{100}{\gram} que requiere \SI{868}{\cal} para elevar su temperatura de \SI{50}{\degreeCelsius} a \SI{90}{\degreeCelsius}. Consulta el valor obtenido en la tabla de la presentación de clase, a fin de identificar de qué sustancia se trata.
\item Determina la cantidad de calor que cede al ambiente una barra de plata de \SI{600}{\gram} al enfriarse de \SI{200}{\degreeCelsius} a \SI{50}{\degreeCelsius}.




\end{enumerate}


\end{document}