\documentclass[14pt]{beamer}
\usepackage{./Estilos/BeamerUVM}
\usepackage{./Estilos/ColoresLatex}
\input{Preambulos/preambulo_Beamer_Cambridge_wolverine}
% \usefonttheme{serif}
\usepackage[clock]{ifsym}
\DeclareSIUnit\erg{erg}
\DeclareSIUnit[number-unit-product = {\,}]\cal{cal}

\sisetup{per-mode=symbol}
\resetcounteronoverlays{saveenumi}

\title{\Large{Electricidad y Magnetismo} \\ \normalsize{Física 2}}
\date{24 de julio de 2023}

\begin{document}
\maketitle

\section*{Contenido}
\frame[allowframebreaks]{\frametitle{Contenido} \tableofcontents[currentsection, hideallsubsections]}

\section{Electricidad}
\frame{\tableofcontents[currentsection, hideothersubsections]}
\subsection{Introducción}

\begin{frame}
\frametitle{Preguntas disparadoras}
¿Te has imaginado vivir en un mundo sin energía eléctrica?
\\
\bigskip
\pause
Esto implicaría no usar aparatos electrónicos como radio, televisión, grabadora y otros más.
\end{frame}
\begin{frame}
\frametitle{Nooooo!}
\begin{figure}
    \centering
    \includegraphics[scale=0.1]{Imagenes/emoji_sorpresa.jpg}
\end{figure}
\end{frame}
\begin{frame}
\frametitle{Aprovechamiento de la electricidad}
Por lo tanto, \pause contar con un tipo de fuente de energía, empleando una propiedad física, \pause nos facilita y mejora nuestra calidad de vida, porque sin ella, no contaríamos con iluminación
y calor.
\end{frame}
\begin{frame}
\frametitle{Aprovechamiento de la electricidad}
Esto significa que con esta fuente de energía se ponen en marcha diferentes tipos de máquinas, artefactos y sistemas de transporte, por mencionar algunos.
\end{frame}
\begin{frame}
\frametitle{¿Qué es la electricidad?}
La palabra electricidad se deriva de la raíz griega elektron, que significa ámbar.
\end{frame}
\begin{frame}
\frametitle{¿Qué es la electricidad?}
La electricidad se define como un fenómeno físico que se origina del \textocolor{flame}{movimiento de partículas subatómicas por medio de cargas eléctricas} a través de la atracción y repulsión de las mismas.
\end{frame}
\begin{frame}
\frametitle{¿Qué es la electricidad?}
la Electricidad es una rama de la Física que estudia todos los fenómenos relacionados con las cargas eléctricas en reposo o movimiento.
\end{frame}
\begin{frame}
\frametitle{Electrostática}
Rama de la electricidad que se encarga de estudiar las \textocolor{folly}{cargas eléctricas en reposo}.
\end{frame}
\begin{frame}
\frametitle{Electrodinámica}
Es la rama de la electricidad que se encarga de estudiar las \textocolor{halayaube}{cargas eléctricas en movimiento}.
\end{frame}
\begin{frame}
\frametitle{Propiedad fundamental}
La \textocolor{hanpurple}{carga eléctrica} es una propiedad fundamental de la materia y base de todos los fenómenos de interacción eléctrica.
\\
\bigskip
\pause
Se representa con la letra $q$.
\end{frame}
\begin{frame}
\frametitle{Tipos de carga eléctrica}
Las cargas eléctricas son de dos tipos:
\pause
\begin{figure}
    \centering
    \begin{tikzpicture}
        \node at (0, 0) {Tipos de carga};
        \draw node at (4.75, 1) {Carga positiva};
        \draw [-stealth] (2, 0) -- (3, 1);
        \draw [fill, color=ao, text=white] (7.5, 1) circle (10pt) node {$+$};

        \draw node at (4.75, -1) {Carga negativa};
        \draw [-stealth] (2, 0) -- (3, -1);
        \draw [fill, color=red, text=white] (7.5, -1) circle (10pt) node {$-$};
        
    \end{tikzpicture}
\end{figure}
\end{frame}
\end{document}