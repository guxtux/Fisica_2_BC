\documentclass[14pt]{extarticle}
\usepackage[utf8]{inputenc}
\usepackage[T1]{fontenc}
\usepackage[spanish,es-lcroman]{babel}
\usepackage{amsmath}
\usepackage{amsthm}
\usepackage{physics}
\usepackage{tikz}
\usepackage{float}
\usepackage[autostyle,spanish=mexican]{csquotes}
\usepackage[per-mode=symbol]{siunitx}
\usepackage{gensymb}
\usepackage{multicol}
\usepackage{enumitem}
\usepackage[left=2.00cm, right=2.00cm, top=2.00cm, 
     bottom=2.00cm]{geometry}
\usepackage{Estilos/ColoresLatex}
\usepackage{makecell}

\newcommand{\textocolor}[2]{\textbf{\textcolor{#1}{#2}}}
\DeclareSIUnit[number-unit-product = {\,}]\cal{cal}

%\renewcommand{\questionlabel}{\thequestion)}
\decimalpoint
\sisetup{bracket-numbers = false}

\title{\vspace*{-2cm} Ejercicios Electricidad - Física 2\vspace{-5ex}}
\date{\today}

\begin{document}
\maketitle

\section{Ejercicios de Evaluación Continua.}

Resuelve de manera detallada los siguientes ejercicios.
\par
Esta actividad te otorgará hasta $5$ puntos a la Evaluación Continua.
\par
La entrega se hará en impreso, es decir, de manera física e individual, teniendo como plazo el día lunes 31 de julio entrando a la clase de las 13 pm.
\par
Cada ejercicio vale $1$ punto, siempre y cuando esté correcto, se indica que se requiere de un manejo completo y claro de las unidades en cada ejercicio, en caso de que no se tenga este manejo, no aportará puntaje.
\par
Anota en la hoja tu nombre completo, así como una identificación de cada ejercicio.

\begin{enumerate}
\item Una carga de \SI{10}{\micro\coulomb} se encuentra en el aire con otra carga de $-\SI{5}{\micro\coulomb}$. Determina la magnitud de la fuerza eléctrica entre las cargas cuando están separadas \SI{50}{\centi\meter}.
\item Dos cargas iguales de $q_{2} = \SI{50}{\nano\coulomb}$ se encuentran separadas por una distancia de \SI{10}{\milli\meter} en el vacío. ¿Cuál es el valor de la fuerza electrostática?
\item Determina la separación que debe haber entre dos cargas cuya magnitudes son $q_{1} = \SI{8}{\micro\coulomb}$ y $q_{2} = \SI{12}{\micro\coulomb}$, si la fuerza de repulsión en el vacío producida por las cargas es de \SI{0.8}{\newton}.
\item Dos cargas idénticas experimentan una fuerza de repulsión entre ellas de \SI{0.08}{\newton} cuando están separadas en el vacío por una distancia de \SI{40}{\centi\meter}. ¿Cuál es el valor de las cargas?
\item Calcula la magnitud de la fuerza entre dos protones que se encuentran a una distancia de \SI{8.3d-12}{\meter} en el aire.
\end{enumerate}


\end{document}