\documentclass[14pt]{beamer}
\usepackage{./Estilos/BeamerUVM}
\usepackage{./Estilos/ColoresLatex}
%Sección para el tema de beamer, con el theme, usercolortheme y sección de footers
\usetheme{CambridgeUS}
\usecolortheme{default}
%\useoutertheme{default}
\setbeamercovered{invisible}
% or whatever (possibly just delete it)
\setbeamertemplate{section in toc}[sections numbered]
\setbeamertemplate{subsection in toc}[subsections numbered]
\setbeamertemplate{subsection in toc}{\leavevmode\leftskip=3.2em\rlap{\hskip-2em\inserttocsectionnumber.\inserttocsubsectionnumber}\inserttocsubsection\par}
\setbeamercolor{section in toc}{fg=blue}
\setbeamercolor{subsection in toc}{fg=blue}
\setbeamercolor{frametitle}{fg=blue}
\setbeamertemplate{caption}[numbered]

\setbeamertemplate{footline}
\beamertemplatenavigationsymbolsempty
\setbeamertemplate{headline}{}


\makeatletter
\setbeamercolor{secºtion in foot}{bg=gray!30, fg=black!90!orange}
\setbeamercolor{subsection in foot}{bg=blue!30!yellow, fg=red}
\setbeamercolor{date in foot}{bg=black, fg=white}
\setbeamertemplate{footline}
{
  \leavevmode%
  \hbox{%
  \begin{beamercolorbox}[wd=.333333\paperwidth,ht=2.25ex,dp=1ex,center]{section in foot}%
    \usebeamerfont{section in foot} \insertsection
  \end{beamercolorbox}%
  \begin{beamercolorbox}[wd=.333333\paperwidth,ht=2.25ex,dp=1ex,center]{subsection in foot}%
    \usebeamerfont{subsection in foot}  \insertsubsection
  \end{beamercolorbox}%
  \begin{beamercolorbox}[wd=.333333\paperwidth,ht=2.25ex,dp=1ex,right]{date in head/foot}%
    \usebeamerfont{date in head/foot} \insertshortdate{} \hspace*{2em}
    \insertframenumber{} / \inserttotalframenumber \hspace*{2ex} 
  \end{beamercolorbox}}%
  \vskip0pt%
}






% \usefonttheme{serif}
\usepackage[clock]{ifsym}
\DeclareSIUnit\erg{erg}
\DeclareSIUnit[number-unit-product = {\,}]\cal{cal}

\sisetup{per-mode=symbol}
\resetcounteronoverlays{saveenumi}

\title{\Large{Práctica 4 - Calor y temperatura} \\ \normalsize{Física 2}}
\date{6 de julio de 2023}

\begin{document}
\maketitle

\section*{Contenido}
\frame{\frametitle{Contenido} \tableofcontents[currentsection, hideallsubsections]}

\section{La Práctica}
\frame{\tableofcontents[currentsection, hideothersubsections]}
\subsection{Objetivo}

\begin{frame}
\frametitle{Objetivo de la Práctica}
Determinar la curva de ascenso y descenso de temperatura de dos líquidos.
\end{frame}
\begin{frame}
\frametitle{Hipótesis}
Planteamos como hipótesis:
\\
\pause
La sustancia combinada con otras sustancias, tiene una curva de ascenso de temperatura más prolongada que la de una sustancia pura.
\end{frame}

\subsection{Material}

\begin{frame}
\frametitle{Material y equipo}
\setbeamercolor{item projected}{bg=bananayellow,fg=ao}
\setbeamertemplate{enumerate items}{%
\usebeamercolor[bg]{item projected}%
\raisebox{1.5pt}{\colorbox{bg}{\color{fg}\footnotesize\insertenumlabel}}%
}
\begin{enumerate}[<+->]
\item Parrilla eléctrica.
\item 2 vasos de precipitado.
\item 2 termómetros.
\item 2 rejillas de asbesto.
\item Cronómetro (del celular)
\item Agua.
\item Sal.
\end{enumerate}
\end{frame}

\subsection{Procedimiento}

\begin{frame}
\frametitle{Iniciando la práctica}
\setbeamercolor{item projected}{bg=red,fg=white}
\setbeamertemplate{enumerate items}{%
\usebeamercolor[bg]{item projected}%
\raisebox{1.5pt}{\colorbox{bg}{\color{fg}\footnotesize\insertenumlabel}}%
}
\begin{enumerate}[<+->]
\item Coloca \SI{60}{\milli\liter} de agua cada vaso de precipitado.
\item Coloca el termómetro en el vaso y registra la temperatura del agua.
\item Al segundo vaso, agrega sal hasta tener una solución sobresaturada.
\seti
\end{enumerate}
\end{frame}
\begin{frame}
\frametitle{Registrando los valores}
\setbeamercolor{item projected}{bg=red,fg=white}
\setbeamertemplate{enumerate items}{%
\usebeamercolor[bg]{item projected}%
\raisebox{1.5pt}{\colorbox{bg}{\color{fg}\footnotesize\insertenumlabel}}%
}
\begin{enumerate}[<+->]
\conti
\item Coloca el primer vaso en la parrilla.
\item Enciende la parrilla y deja la perilla en la máxima posición.
\seti
\end{enumerate}
\end{frame}
\begin{frame}
\frametitle{Registrando los valores}
\setbeamercolor{item projected}{bg=red,fg=white}
\setbeamertemplate{enumerate items}{%
\usebeamercolor[bg]{item projected}%
\raisebox{1.5pt}{\colorbox{bg}{\color{fg}\footnotesize\insertenumlabel}}%
}
\begin{enumerate}[<+->]
\conti
\item Registra en tu cuaderno el tiempo (en segundos) y el valor de temperatura (grados centígrados) \textocolor{carmine}{CADA 15 SEGUNDOS}.
\item En todo momento el termómetro debe de seguir dentro del vaso.
\item Una vez que haya comenzado a ebullir el agua, retira con cuidado el vaso y colócalo en la rejilla de asbesto.
\seti
\end{enumerate}
\end{frame}
\begin{frame}
\frametitle{Registrando los valores}
\setbeamercolor{item projected}{bg=red,fg=white}
\setbeamertemplate{enumerate items}{%
\usebeamercolor[bg]{item projected}%
\raisebox{1.5pt}{\colorbox{bg}{\color{fg}\footnotesize\insertenumlabel}}%
}
\begin{enumerate}[<+->]
\conti
\item Ahora deberás de registrar el tiempo y la temperatura de descenso \textocolor{cobalt}{CADA 15 SEGUNDOS}.
\item El termómetro no debe de salir del vaso.
\item Repite el mismo procedimiento para el vaso con la solución salina sobresaturada.
\end{enumerate}
\end{frame}

\subsection{Análisis de los datos}

\begin{frame}
\frametitle{Análisis de los datos 1/2}
Deberás de graficar en un mismo espacio las dos curvas de ascenso de temperatura, diferenciando ya sea con un color o una marca.
\\
\bigskip
\pause
No unas los puntos experimentales, solo deben de estar las marcas de tiempo y temperatura.
\end{frame}
\begin{frame}
\frametitle{Graficando los datos experimentales}
\begin{figure}
    \centering
    \begin{tikzpicture}
        \draw [-stealth] (0, 0) -- (5, 0) node [above, pos=1] {$t \, [\unit{\second}]$};
        \draw [-stealth] (0, 0) -- (0, 5) node [left, pos=1] {$T \, [\unit{\degreeCelsius}]$};
    \end{tikzpicture}
\end{figure}
\end{frame}
\begin{frame}
\frametitle{Análisis de los datos 2/2}
Ahora grafica en un mismo espacio las dos curvas de descenso de temperatura, diferenciando ya sea con un color o una marca.
\\
\bigskip
\pause
No unas los puntos experimentales, solo deben de estar las marcas de tiempo y temperatura.
\end{frame}
\begin{frame}
\frametitle{Interpretación 1/2}
Responde las siguientes preguntas:
\pause
\setbeamercolor{item projected}{bg=black,fg=white}
\setbeamertemplate{enumerate items}{%
\usebeamercolor[bg]{item projected}%
\raisebox{1.5pt}{\colorbox{bg}{\color{fg}\footnotesize\insertenumlabel}}%
}
\begin{enumerate}[<+->]
\item En el ascenso de temperatura ¿qué sustancia es la que muestra la curva más prolongada?
\item ¿Cómo explicas esto?
\item ¿El punto de ebullición es el mismo en ambas sustancias?
\item Deberás de apoyarte con una revisión en libros, NO SE ACEPTAN enunciados de AI.
\end{enumerate}
\end{frame}
\begin{frame}
\frametitle{Interpretación 1/2}
Responde las siguientes preguntas:
\pause
\setbeamercolor{item projected}{bg=black,fg=white}
\setbeamertemplate{enumerate items}{%
\usebeamercolor[bg]{item projected}%
\raisebox{1.5pt}{\colorbox{bg}{\color{fg}\footnotesize\insertenumlabel}}%
}
\begin{enumerate}[<+->]
\item En el descenso de temperatura ¿qué sustancia es la que muestra la curva más \enquote{rápida}?
\item ¿Cómo explicas esto?
\item ¿La temperatura ambiente tiene que ver con estos resultados?
\item Deberás de apoyarte con una revisión en libros, NO SE ACEPTAN enunciados de AI.
\end{enumerate}
\end{frame}

\subsection{El reporte}

\begin{frame}
\frametitle{Envío del reporte}
Una vez que concluyas la interpretación y explicaciones señaladas, señalar si se cumplió el objetivo de la práctica y si la hipótesis inicial fue correcta, \pause deberás de enviar tu reporte INDIVIDUAL en la asignación de Teams.
\end{frame}
\begin{frame}
\frametitle{Fecha de entrega}
Recuerda incluir tu nombre y la autoevaluación que se indica en el Manual de Prácticas.
\\
\bigskip
\pause
Tu informe deberá de enviarse a más tardar:
\begin{itemize}
\item Grupo 41C (clase del martes) El lunes 10 de julio a las 8 pm.
\item Grupo 41B (clase del jueves) El miércoles 12 de julio a las 8 pm.
\end{itemize}
\end{frame}
\end{document}