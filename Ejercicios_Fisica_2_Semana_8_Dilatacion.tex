\documentclass[14pt]{extarticle}
\usepackage[utf8]{inputenc}
\usepackage[T1]{fontenc}
\usepackage[spanish,es-lcroman]{babel}
\usepackage{amsmath}
\usepackage{amsthm}
\usepackage{physics}
\usepackage{tikz}
\usepackage{float}
\usepackage[autostyle,spanish=mexican]{csquotes}
\usepackage[per-mode=symbol]{siunitx}
\usepackage{gensymb}
\usepackage{multicol}
\usepackage{enumitem}
\usepackage[left=2.00cm, right=2.00cm, top=2.00cm, 
     bottom=2.00cm]{geometry}
\usepackage{Estilos/ColoresLatex}
\usepackage{makecell}

\newcommand{\textocolor}[2]{\textbf{\textcolor{#1}{#2}}}
\DeclareSIUnit[number-unit-product = {\,}]\cal{cal}

%\renewcommand{\questionlabel}{\thequestion)}
\decimalpoint
\sisetup{bracket-numbers = false}

\title{\vspace*{-2cm} Ejercicios Dilatación Térmica - Física 2\vspace{-5ex}}
\date{\today}

\begin{document}
\maketitle

\section{Ejercicios a cuenta}

Resuelve de manera detallada los siguientes ejercicios.

Esta actividad te otorgará hasta  $5$ puntos a la Evaluación Continua.

La entrega se hará vía Teams en asignación, teniendo como plazo el día domingo 16 de julio a las 8 pm.

Cada ejercicio vale $1$ punto, siempre y cuando esté correcto. Se otorgará una parte proporcional en caso de tener desarrollo detallado, pero el resultado no sea el esperado.

Anota en la hoja tu nombre completo, así como una identificación de cada ejercicio.

\begin{enumerate}
\item La longitud de un puente de concreto es de \SI{1}{\kilo\meter} a una temperatura de \SI{20}{\degreeCelsius}. ¿Cuál es la longitud cuando la temperatura es de \SI{38}{\degreeCelsius}?
\item Se tienen 3 barras de diferentes materiales a una misma temperatura inicial, realiza las operaciones necesarias y completa la tabla que se muestra a continuación.
\begin{table}[H]
\centering
\begin{tabular}{c | c | c | c | c | c}
Barra & \makecell{Long. inicial \\ (\unit{\meter})} & \makecell{Temp. inicial \\ (\unit{\degreeCelsius})} & \makecell{Temp. final \\ (\unit{\degreeCelsius})} & \makecell{Long. final \\ (\unit{\meter})} & Dilatación \\ \hline
Acero & $10$ & $20$ & $60$ & & \\ \hline
Hierro & $10$ & $20$ & $60$ & & \\ \hline
Aluminio & $10$ & $20$ & $60$ & & \\ \hline    
\end{tabular}
\end{table}
¿Cuál barra se dilató más? ¿Por qué? Justifica tu respuesta.
\item ¿Cuál es la longitud de un cable de cobre al disminuir la temperatura a \SI{14}{\degreeCelsius}, si con una temperatura de \SI{42}{\degreeCelsius} mide \SI{416}{\meter}?
% \item Un puente de acero de \SI{100}{\meter} de largo a \SI{8}{\degreeCelsius}, aumenta su temperatura a \SI{24}{\degreeCelsius}. ¿Cuánto medirá su longitud?
% \item La temperatura inicial de una barra de aluminio de \SI{3}{\kilo\gram} es de \SI{25}{\degreeCelsius}. ¿Cuál será su temperatura final si al ser calentada recibe \SI{12000}{\cal}?
\item ¿Cuántas calorías se deben suministrar para que un trozo de hierro de \SI{0.3}{\kilo\gram} eleve su temperatura de \SI{20}{\degreeCelsius} a \SI{100}{\degreeCelsius}?
\item Determina el calor específico de una muestra metálica de \SI{400}{\gram} si al suministrarle \SI{620}{\cal} aumentó su temperatura de \SI{15}{\degreeCelsius} a \SI{65}{\degreeCelsius}. Consulta en tablas el valor obtenido e identifica de qué sustancia se trata.
\end{enumerate}


\end{document}