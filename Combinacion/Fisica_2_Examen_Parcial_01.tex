\documentclass[14pt]{extarticle}
\usepackage[utf8]{inputenc}
\usepackage[T1]{fontenc}
\usepackage[spanish,es-lcroman]{babel}
\usepackage{amsmath}
\usepackage{amsthm}
\usepackage{physics}
\usepackage{tikz}
\usepackage{float}
\usepackage[autostyle,spanish=mexican]{csquotes}
\usepackage[per-mode=symbol]{siunitx}
\usepackage{gensymb}
\usepackage{multicol}
\usepackage{enumitem}
\usepackage[left=2.00cm, right=2.00cm, top=2.00cm, 
     bottom=2.00cm]{geometry}

%\renewcommand{\questionlabel}{\thequestion)}
\decimalpoint
\sisetup{bracket-numbers = false}

\title{\vspace*{-2cm} Primer examen parcial (Examen A)\\  Curso de Física 2\vspace{-5ex}}
\date{\today}

\begin{document}
\maketitle

\begin{enumerate}
\item Considerando los estados de agregación de la materia, el cambio de estado:
\begin{enumerate}[label=\Roman*)]
\item Sólido a Líquido.
\item Sólido a Gas.
\item Gas a Líquido.
\end{enumerate}
Se llaman:

R = I) Fusión, II) Sublimación, III) Condensación

\item ¿Cuál de los siguientes incisos NO define a un fluido?

R = Mantienen una forma fija.

\item Se enfoca en el análisis de las propiedades de los fluidos sin considerar las fuerzas asociadas al movimiento, hablamos de:

R = Hidrostática.

\item Calcula la presión que ejerce una mujer de \SI{70}{\kilo\gram} sobre sus pies, al estar de pie sobre una superficie horizontal de \SI{0.067}{\square\meter}.

R = \SI{1.02492d4}{\pascal}

\item A continuación se presentan tres conceptos y cuatro definiciones, selecciona la respuesta que relacione el concepto y la definición.
\begin{enumerate}[label=\arabic*)]
\item Peso específico.
\item Cohesión.
\item Densidad.
\end{enumerate}
\begin{enumerate}[label=\alph*)]
\item Es la fuerza que actúa en la superficie de un líquido y disminuye su área superficial.
\item Es la fuerza de atracción entre moléculas del mismo tipo en un líquido.
\item Es el peso de un fluido por unidad de volumen.
\item Es la relación entre la masa y el volumen.
\end{enumerate}

R = 1) - c), 2) - b), 3) - d)

\item ¿Cuál es el volumen ocupado por \SI{500}{\gram} de vidrio, si la densidad del vidrio es \SI{2.6}{\gram\per\cubic\metre}?

R = \SI{1.923d-4}{\cubic\meter}

\item La presión absoluta que existe en un recipiente cerrado es igual a la suma de la presión manométrica más la:

R = Presión absoluta.

\item 

\end{enumerate}


\end{document}