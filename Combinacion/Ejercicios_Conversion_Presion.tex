\documentclass[14pt]{beamer}
\usepackage{Estilos/BeamerUVM}
\usepackage{Estilos/ColoresLatex}
\usetheme{Madrid}
\usecolortheme{default}
%\useoutertheme{default}
\setbeamercovered{invisible}
% or whatever (possibly just delete it)
\setbeamertemplate{section in toc}[sections numbered]
\setbeamertemplate{subsection in toc}[subsections numbered]
\setbeamertemplate{subsection in toc}{\leavevmode\leftskip=3.2em\rlap{\hskip-2em\inserttocsectionnumber.\inserttocsubsectionnumber}\inserttocsubsection\par}
% \setbeamercolor{section in toc}{fg=blue}
% \setbeamercolor{subsection in toc}{fg=blue}
% \setbeamercolor{frametitle}{fg=blue}
\setbeamertemplate{caption}[numbered]

\setbeamertemplate{footline}
\beamertemplatenavigationsymbolsempty
\setbeamertemplate{headline}{}


\makeatletter
% \setbeamercolor{section in foot}{bg=gray!30, fg=black!90!orange}
% \setbeamercolor{subsection in foot}{bg=blue!30}
% \setbeamercolor{date in foot}{bg=black}
\setbeamertemplate{footline}
{
  \leavevmode%
  \hbox{%
  \begin{beamercolorbox}[wd=.333333\paperwidth,ht=2.25ex,dp=1ex,center]{section in foot}%
    \usebeamerfont{section in foot} {\insertsection}
  \end{beamercolorbox}%
  \begin{beamercolorbox}[wd=.333333\paperwidth,ht=2.25ex,dp=1ex,center]{subsection in foot}%
    \usebeamerfont{subsection in foot}  \insertsubsection
  \end{beamercolorbox}%
  \begin{beamercolorbox}[wd=.333333\paperwidth,ht=2.25ex,dp=1ex,right]{date in head/foot}%
    \usebeamerfont{date in head/foot} \insertshortdate{} \hspace*{2em}
    \insertframenumber{} / \inserttotalframenumber \hspace*{2ex} 
  \end{beamercolorbox}}%
  \vskip0pt%
}
\makeatother

\makeatletter
\patchcmd{\beamer@sectionintoc}{\vskip1.5em}{\vskip0.8em}{}{}
\makeatother

% \usefonttheme{serif}
\usepackage[clock]{ifsym}

\sisetup{per-mode=symbol}
\DeclareSIUnit\atm{atm}
\resetcounteronoverlays{saveenumi}

\title{\Large{Conversión de Unidades de Presión} \\ \normalsize{Física 2}}
\date{14 de junio de 2023}

\begin{document}
\maketitle

\section*{Contenido}
\frame{\frametitle{Contenido} \tableofcontents[currentsection, hideallsubsections]}


\section{Conversión de unidades}
\frame{\tableofcontents[currentsection, hideothersubsections]}
\subsection{Base principal}

\begin{frame}
\frametitle{La conversión de unidades}
En física es muy común realizar cambios de unidades para alguna magnitud tanto en el mismo sistema de medición, como en otros sistemas.
\\
\bigskip
\pause
La conversión de unidades es un procedimiento sencillo que se realiza de manera sencilla.
\end{frame}
\begin{frame}
\frametitle{Los factores de conversión}
El \textocolor{cobalt}{primer paso} es: \pause contar con el (los) factor(es) de conversión, de esta manera reducimos el trabajo y nos anticipamos favorablemente para resolver el ejercicio.
\end{frame}
\begin{frame}
\frametitle{Factores directos e intermedios}
En algunos ejercicios será posible hacer la conversión ocupando un solo factor de conversión, \pause a esta conversión le llamaremos \textocolor{red}{conversión directa}.
\\
\bigskip
\pause
Por ejemplo, convertir de kilómetros a metros.
\end{frame}
\begin{frame}
\frametitle{Factores directos e intermedios}
En el caso de que se requiera de otros factores de conversión adicionales, le llamaremos \textocolor{magenta}{conversión intermedia}.
\\
\bigskip
\pause
Deberán de ocuparse los factores de conversión necesarios para llegar a las unidades que se solicitaron.
\end{frame}
\begin{frame}
\frametitle{Siguiente paso}
El \textocolor{brickred}{segundo paso} consiste en multplicar la cantidad incial a convertir por cada uno de los factores de conversión.
\\
\bigskip
\pause
Revisa con cuidado la manera en que presentas el factor de conversión, ya que debes de garantizar que las unidades iniciales se \enquote{cancelen}.
\end{frame}
\begin{frame}
\frametitle{Cancelando las unidades}
Convertir \SI{64}{\kilo\meter\per\hour} a \unit{\meter\per\second}
\\
\bigskip
\pause
Anotamos previamente los factores de conversión:
\begin{align*}
\SI{1}{\kilo\meter} &= \SI{1000}{\meter} \\[0.5em]
1 \, \, \text{hora} &= \SI{60}{\minute} \times \SI{60}{\second} = \SI{3600}{\second}
\end{align*}
\end{frame}
\begin{frame}
\frametitle{Convirtiendo las unidades}
Presentamos la conversión:
\pause
\begin{eqnarray*}
\begin{aligned}
\SI[per-mode=fraction]{64}{\kilo\meter\per\hour} \times \dfrac{\SI{1000}{\meter}}{\SI{1}{\kilo\meter}} \times \dfrac{\SI{1}{\hour}}{\SI{3600}{\second}} =
\end{aligned}
\end{eqnarray*}
\end{frame}
\begin{frame}
\frametitle{Convirtiendo las unidades}
Presentamos la conversión:
\begin{eqnarray*}
\begin{aligned}
&64 \, \dfrac{\Cancel[red]{\text{km}}}{\Cancel[red]{\text{h}}} \times \dfrac{\SI{1000}{\meter}}{\SI{1}{\Cancel[red]{\kilo\meter}}} \times \dfrac{\SI{1}{\Cancel[red]{\hour}}}{\SI{3600}{\second}} = \dfrac{\SI{6.41d4}{\meter}}{\SI{3.6d3}{\second}} = \\[0.5em] \pause
&= \left( \dfrac{6.41}{3.6} \right) \times 10^{4-3} = \pause \SI[per-mode=fraction]{17.8}{\meter\per\second}
\end{aligned}
\end{eqnarray*}
\end{frame}


\section{Unidades de Presión}
\frame{\tableofcontents[currentsection, hideothersubsections]}
\subsection{Ejercicios de conversión}

\begin{frame}
\frametitle{Ejercicio de conversión}
Realiza la conversión de \SI{1}{\atm} a dinas/\unit{\square\centi\meter}
\\
\bigskip
\pause
Para identificar los factores de conversión, hay que recordar varias definiciones.
\end{frame}
\begin{frame}
\frametitle{Definiciones previas}
Recordemos que:
\begin{eqnarray*}
\begin{aligned}
\SI{1}{atm} &= \SI{1.01325d5}{\pascal} \\[0.5em] \pause
\SI{1}{\pascal} &= \SI[per-mode=fraction]{1}{\newton\per\square\meter} \\[0.5em] \pause
\SI{1}{\newton} &= \SI[per-mode=fraction]{1}{\kilo\gram\meter\per\square\second} \\[0.5em] \pause
1 \,\, \text{dina} &= \SI[per-mode=fraction]{1}{\gram\centi\meter\per\square\second}
\end{aligned}
\end{eqnarray*}
\end{frame}
\begin{frame}
\frametitle{La conversión a realizar}
Entonces lo que nos piden para la conversión es:
\pause
\begin{eqnarray*}
\begin{aligned}
\num{1.01325d5} \dfrac{\displaystyle \unit[per-mode=fraction]{\kilo\gram\meter\per\square\second}}{\unit{\square\meter}} \quad \rightarrow \quad \dfrac{\displaystyle \unit[per-mode=fraction]{\gram\centi\meter\per\square\second}}{\unit{\square\centi\meter}}
\end{aligned}
\end{eqnarray*}
\end{frame}
\begin{frame}
\frametitle{Los factores de conversión}
Vamos a ocupar los siguientes factores de conversión:
\pause
\begin{eqnarray*}
\begin{aligned}
\SI{1}{\kilo\gram} &= \SI{d3}{\gram} \\[0.5em] \pause
\SI{1}{\meter} &= \SI{d2}{\centi\meter} \\[0.5em] \pause
\left( \SI{1}{\meter} \right)^{2} &= \left( \SI{d2}{\centi\meter} \right)^{2} \\[0.5em] \pause
\SI{1}{\square\meter} &= \SI{d4}{\square\centi\meter}
\end{aligned}
\end{eqnarray*}
\end{frame}
\begin{frame}
\frametitle{Ajustando la expresión}
Para simplificar la operación, hagamos el siguiente ajuste:
\pause
\begin{eqnarray*}
\begin{aligned}
\num{1.01325d5} \dfrac{\displaystyle \unit[per-mode=fraction]{\kilo\gram\meter\per\square\second}}{\unit{\square\meter}} = \num{1.01325d5} \, \, \dfrac{\unit{\kilo\gram\meter}}{\unit{\square\second} \unit{\square\meter}}
\end{aligned}
\end{eqnarray*}
\end{frame}
\begin{frame}
\frametitle{Usando los factores de conversión}
\begin{eqnarray*}
\begin{aligned}
&\num{1.01325d5} \, \, \dfrac{\unit{\kilo\gram\meter}}{\unit{\square\second} \unit{\square\meter}} \left( \dfrac{\SI{d3}{\gram}}{\SI{1}{\kilo\gram}} \right) \left( \dfrac{\SI{d2}{\centi\meter}}{\SI{1}{\meter}} \right) \left( \dfrac{\SI{1}{\square\meter}}{\SI{d4}{\square\centi\meter}} \right) = \\[0.5em] \pause
&= \dfrac{\SI{1.01325d10}{\gram\centi\meter}}{\num{d4} \, \, \unit{\square\second} \unit{\square{\centi\meter}}} =
\end{aligned}
\end{eqnarray*}

\end{frame}

\end{document}