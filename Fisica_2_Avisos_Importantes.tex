\documentclass[14pt]{beamer}
\usepackage{./Estilos/BeamerUVM}
\usepackage{./Estilos/ColoresLatex}
\usetheme{Madrid}
\usecolortheme{default}
%\useoutertheme{default}
\setbeamercovered{invisible}
% or whatever (possibly just delete it)
\setbeamertemplate{section in toc}[sections numbered]
\setbeamertemplate{subsection in toc}[subsections numbered]
\setbeamertemplate{subsection in toc}{\leavevmode\leftskip=3.2em\rlap{\hskip-2em\inserttocsectionnumber.\inserttocsubsectionnumber}\inserttocsubsection\par}
% \setbeamercolor{section in toc}{fg=blue}
% \setbeamercolor{subsection in toc}{fg=blue}
% \setbeamercolor{frametitle}{fg=blue}
\setbeamertemplate{caption}[numbered]

\setbeamertemplate{footline}
\beamertemplatenavigationsymbolsempty
\setbeamertemplate{headline}{}


\makeatletter
% \setbeamercolor{section in foot}{bg=gray!30, fg=black!90!orange}
% \setbeamercolor{subsection in foot}{bg=blue!30}
% \setbeamercolor{date in foot}{bg=black}
\setbeamertemplate{footline}
{
  \leavevmode%
  \hbox{%
  \begin{beamercolorbox}[wd=.333333\paperwidth,ht=2.25ex,dp=1ex,center]{section in foot}%
    \usebeamerfont{section in foot} {\insertsection}
  \end{beamercolorbox}%
  \begin{beamercolorbox}[wd=.333333\paperwidth,ht=2.25ex,dp=1ex,center]{subsection in foot}%
    \usebeamerfont{subsection in foot}  \insertsubsection
  \end{beamercolorbox}%
  \begin{beamercolorbox}[wd=.333333\paperwidth,ht=2.25ex,dp=1ex,right]{date in head/foot}%
    \usebeamerfont{date in head/foot} \insertshortdate{} \hspace*{2em}
    \insertframenumber{} / \inserttotalframenumber \hspace*{2ex} 
  \end{beamercolorbox}}%
  \vskip0pt%
}
\makeatother

\makeatletter
\patchcmd{\beamer@sectionintoc}{\vskip1.5em}{\vskip0.8em}{}{}
\makeatother

% \usefonttheme{serif}
\usepackage[clock]{ifsym}

\sisetup{per-mode=symbol}
\resetcounteronoverlays{saveenumi}

\title{\Large{Avisos importantes} \\ \normalsize{Física 2}}
\date{26 de junio de 2023}

\begin{document}
\maketitle

\section*{Contenido}
\frame{\frametitle{Contenido} \tableofcontents[currentsection, hideallsubsections]}

\section{Avisos importantes}
\frame{\tableofcontents[currentsection, hideothersubsections]}
\subsection{En las clases}

\begin{frame}
\frametitle{Regreso a las clases presenciales}
Se debe de mantener en todo momento dentro del Campus San Rafael, un orden, disciplina y respeto.
\\
\bigskip
\pause
Hay que mantenerse atentos a la clase, sin distracciones o generando ruido.
\end{frame}
\begin{frame}
\frametitle{Pase de asistencia}
Como norma dentro de la clase, se llevará a cabo el Pase de Asistencia a los 6 minutos de iniciada la clase.
\\
\bigskip
\pause
En caso de ingresar luego de haber nombrado a la alumna/alumno, no se corregirá la inasistencia, a menos que se tenga un justificante por parte de la Coordinación Académica.
\end{frame}
\begin{frame}
\frametitle{Faltas en el período}
Como medida adicional, previo a la evaluación del segundo y tercer examen parcial, se presentará a cada alumno el total de inasistencias.
\end{frame}
\begin{frame}
\frametitle{Seguimiento de actividades}
De manera semanal, el Profesor enviará a cada alumna/alumno la relación de ejercicios de Evaluación Continua y de Prácticas que se hayan dejado en la semana anterior.
\\
\bigskip
\pause
El alumno también llevará su propio registro de actividades, esto nos facilitará el proceso de realizar las calificaciones correspondientes.
\end{frame}
\begin{frame}
\frametitle{Actividades en tiempo y forma}
El envío de actividades de Evaluación Continua y de Laboratorio, se seguirá realizando mediante Teams, con fecha y hora establecida de entrega.
\\
\bigskip
\pause
En caso de que se tenga un problema para enviar su documento, deberá de tomar evidencia del problema, \textocolor{magenta}{solo con evidencia se recibirá la actividad}.
\end{frame}
\begin{frame}
\frametitle{Actividades en tiempo y forma}
Cada actividad de trabajo formará parte de la segunda y tercera evaluación parcial.
\\
\bigskip
\pause
No habrá trabajos adicionales a modo de \enquote{compensar} las actividades no entregadas.
\end{frame}
\begin{frame}
\frametitle{Entrega de actividades}
Todo trabajo deberá ser enviado mediante Teams.
\\
\bigskip
\pause
Si el Profesor requiere que se envíe de nuevo un documento, él mismo solicitará mediante mensaje directo a la alumna/alumno. \pause Todo archivo que se reciba sin ser solicitado, no se atenderá, como tampoco será considerado como evidencia de entrega.
\end{frame}

\section{Laboratorio}
\frame{\tableofcontents[currentsection, hideothersubsections]}
\subsection{Bata blanca}

\begin{frame}
\frametitle{Revisión del reglamento}
Al tener ya sesiones de Laboratorio, en todo momento se observará el Reglamente y las Normas de Seguridad e Higiene.
\\
\bigskip
\pause
Siendo necesario el uso de \textocolor{cerulean}{bata blanca}.
\end{frame}
\begin{frame}
\frametitle{Portar la bata blanca}
Para permanecer en el Laboratorio, se deberá de ingresar con la bata blanca.
\\
\bigskip
\pause
Quien no tenga la bata, no podrá permanecer en la sesión y no tendrá oportunidad de reponer la práctica, menos entregarla.
\end{frame}
\begin{frame}
\frametitle{Espacio de trabajo}
El Profesor organizará equipos de trabajo de tal manera que se tenga el número de integrantes oportuno para trabajar cada práctica.
\end{frame}
\begin{frame}
\frametitle{Práctica en la semana}
Deberán de traer sus resultados de la Práctica 3: el principio de Bernoulli y de Torricelli, para discusión de los resultados, tanto de la parte de la cortina de la Presa La Angostura, como de la parte de experimento en casa.
\end{frame}

\end{document}