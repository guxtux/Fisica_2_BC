\documentclass[14pt]{beamer}
\usepackage{./Estilos/BeamerUVM}
\usepackage{./Estilos/ColoresLatex}
\input{Preambulos/preambulo_Beamer_Cambridge_default}
% \usefonttheme{serif}
\usepackage[clock]{ifsym}
\DeclareSIUnit\erg{erg}
\DeclareSIUnit[number-unit-product = {\,}]\cal{cal}

\sisetup{per-mode=symbol}
\resetcounteronoverlays{saveenumi}

\title{\Large{5 - Fenómenos termodiánmicos} \\ \normalsize{Física 2}}
\date{11 de julio de 2023}

\begin{document}
\maketitle

\section*{Contenido}
\frame{\frametitle{Contenido} \tableofcontents[currentsection, hideallsubsections]}

\section{La Práctica}
\frame{\tableofcontents[currentsection, hideothersubsections]}
\subsection{Objetivo}

\begin{frame}
\frametitle{Objetivo de la Práctica}
Describir en términos de calor y temperatura el cambio de varias sustancias al transfererirles energía calorífica.
\end{frame}

\subsection{Material}

\begin{frame}
\frametitle{Material y equipo}
\setbeamercolor{item projected}{bg=bananayellow,fg=ao}
\setbeamertemplate{enumerate items}{%
\usebeamercolor[bg]{item projected}%
\raisebox{1.5pt}{\colorbox{bg}{\color{fg}\footnotesize\insertenumlabel}}%
}
\begin{enumerate}[<+->]
\item Mechero de alcohol.
\item Soporte universal.
\item Un termómetro.
\item Rejilla de asbesto.
\item Papel alumninio.
\item Cronómetro (del celular)
\seti
\end{enumerate}
\end{frame}
\begin{frame}
\frametitle{Material y equipo}
\setbeamercolor{item projected}{bg=bananayellow,fg=ao}
\setbeamertemplate{enumerate items}{%
\usebeamercolor[bg]{item projected}%
\raisebox{1.5pt}{\colorbox{bg}{\color{fg}\footnotesize\insertenumlabel}}%
}
\begin{enumerate}[<+->]
\conti    
\item Una vela.
\item Un trozo de chocolate.
\item Una báscula.
\item Una rondana.
\end{enumerate}
\end{frame}

\subsection{Procedimiento}

\begin{frame}
\frametitle{Iniciando la práctica}
\setbeamercolor{item projected}{bg=red,fg=white}
\setbeamertemplate{enumerate items}{%
\usebeamercolor[bg]{item projected}%
\raisebox{1.5pt}{\colorbox{bg}{\color{fg}\footnotesize\insertenumlabel}}%
}
\begin{enumerate}[<+->]
\item Recorta tres cuadrados de papel aluminio de $5 \times 5 \, \unit{\centi\meter}$
\item Construye una \enquote{charola} con el cuadrado de papel aluminio.
\item Pesa en la báscula cada sustancia: parafina, trozo de chocolate, rondana.
\seti
\end{enumerate}
\end{frame}
\begin{frame}
\frametitle{Observando y anotando}
\setbeamercolor{item projected}{bg=red,fg=white}
\setbeamertemplate{enumerate items}{%
\usebeamercolor[bg]{item projected}%
\raisebox{1.5pt}{\colorbox{bg}{\color{fg}\footnotesize\insertenumlabel}}%
}
\begin{enumerate}[<+->]
\conti
\item Registra la temperatura del objeto.
\item Con la supervisión del Profesor, enciende el mechero de alcohol.
\item Coloca primero la charola de aluminio con la parafina, inicia el registro de tiempo.
\item Observa y anota detalladamente lo que ocurre a la sustancia, esta parte será la más relevante en tu reporte.
\seti
\end{enumerate}
\end{frame}
\begin{frame}
\frametitle{Observando y anotando}
\setbeamercolor{item projected}{bg=red,fg=white}
\setbeamertemplate{enumerate items}{%
\usebeamercolor[bg]{item projected}%
\raisebox{1.5pt}{\colorbox{bg}{\color{fg}\footnotesize\insertenumlabel}}%
}
\begin{enumerate}[<+->]
\conti
\item Al observar un cambio de fase, registra la temperatura del termómetro y retira la charola y deja que se enfríe.
\item Coloca la siguiente charola de aluminio con el trozo de chocolate.
\item Repite el mismo procedimiento de registro de tiempo y detallar lo que le sucede a la sustancia.
\item Has lo mismo para la charola con la rondana.
\seti
\end{enumerate}
\end{frame}
\begin{frame}
\frametitle{Firma de Trabajo}
El Profesor firmará como evidencia de trabajo en el cuaderno de cada integrante, esto le dará la oportunidad de presentar el reporte de la Práctica.
\end{frame}

\subsection{Análisis de los datos}

\begin{frame}
\frametitle{Análisis de los datos 1/2}
En tu reporte deberás de describir a detalle lo que le ocurrió a cada sustancia, mencionando el peso inicial, a temperatura ambiente, la temperatura en la que se retiró de la rejilla de asbesto.
\end{frame}
\begin{frame}
\frametitle{Análisis de los datos 2/2}
La descripción debe de hacerse en términos de los conceptos que hemos visto en clase de calor y temperatura.
\end{frame}
\begin{frame}
\frametitle{Interpretación}
Responde las siguientes preguntas:
\pause
\setbeamercolor{item projected}{bg=black,fg=white}
\setbeamertemplate{enumerate items}{%
\usebeamercolor[bg]{item projected}%
\raisebox{1.5pt}{\colorbox{bg}{\color{fg}\footnotesize\insertenumlabel}}%
}
\begin{enumerate}[<+->]
\item ¿Qué sustancia cambio de fase en menos tiempo? ¿Por qué?
\item ¿Qué fenómeno termodinámico le sucedió a la rondana?
\seti
\end{enumerate}
\end{frame}
\begin{frame}
\frametitle{Interpretación}
Responde las siguientes preguntas:
\pause
\setbeamercolor{item projected}{bg=black,fg=white}
\setbeamertemplate{enumerate items}{%
\usebeamercolor[bg]{item projected}%
\raisebox{1.5pt}{\colorbox{bg}{\color{fg}\footnotesize\insertenumlabel}}%
}
\begin{enumerate}[<+->]
\conti
\item ¿Por qué al enfriarse las dos primeras sustancias, no se recuperó la forma inicial?
\item Con los datos registrados, ¿podemos reportar el calor específico de la parafina, el chocolate y la rondana?
\item Deberás de apoyarte con una revisión en libros, NO SE ACEPTAN enunciados de AI.
\end{enumerate}
\end{frame}


\subsection{El reporte}

\begin{frame}
\frametitle{Envío del reporte}
Una vez que concluyas la interpretación y explicaciones señaladas, responder si se cumplió el objetivo de la práctica, todo lo anterior, por escrito \pause deberán de reunir la respuesta de cada integrante del equipo, elaborando un solo archivo y enviar cada quien, el reporte en la asignación de Teams.
\end{frame}
\begin{frame}
\frametitle{Envío del reporte}
La aportación individual debe de estar identificada con el nombre de la alumna o alumno.
\\
\bigskip
\pause
El reporte debe de incluir a todos los integrantes, en caso de que falte la aportación de alguna(o) le restará puntuación al trabajo en equipo.
\end{frame}
\begin{frame}
\frametitle{Fecha de entrega}
Recuerda incluir tu nombre y la autoevaluación que se indica en el Manual de Prácticas.
\\
\bigskip
\pause
Tu informe deberá de enviarse a más tardar:
\begin{itemize}
\item Grupo 41C (clase del martes) El lunes 17 de julio a las 8 pm.
\item Grupo 41B (clase del jueves) El miércoles 19 de julio a las 8 pm.
\end{itemize}
\end{frame}
\end{document}