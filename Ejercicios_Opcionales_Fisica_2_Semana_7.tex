\documentclass[14pt]{extarticle}
\usepackage[utf8]{inputenc}
\usepackage[T1]{fontenc}
\usepackage[spanish,es-lcroman]{babel}
\usepackage{amsmath}
\usepackage{amsthm}
\usepackage{physics}
\usepackage{tikz}
\usepackage{float}
\usepackage[autostyle,spanish=mexican]{csquotes}
\usepackage[per-mode=symbol]{siunitx}
\usepackage{gensymb}
\usepackage{multicol}
\usepackage{enumitem}
\usepackage[left=2.00cm, right=2.00cm, top=2.00cm, 
     bottom=2.00cm]{geometry}
\usepackage{Estilos/ColoresLatex}

\newcommand{\textocolor}[2]{\textbf{\textcolor{#1}{#2}}}

%\renewcommand{\questionlabel}{\thequestion)}
\decimalpoint
\sisetup{bracket-numbers = false}

\title{\vspace*{-2cm} Ejercicios Opcionales - Física 2\vspace{-5ex}}
\date{\today}

\begin{document}
\maketitle

\section{Ejercicios a cuenta}

Con la finalidad de apoyar en la recuperación del promedio para los siguientes exámenes parciales, se dejarán una serie de \textocolor{red}{ejercicios adicionales} para Evaluación Continua.


Estos ejercicios serán de carácter \textocolor{cobalt}{opcional}, es decir, la alumna o alumno que desee resolverlos y enviarlos, les sumará $5$ puntos adicionales a la Evaluación Continua.

La entrega se hará vía Teams en asignación, teniendo como plazo el día domingo 9 de julio a las 8 pm.

Cada ejercicio vale $1$ punto, siempre y cuando esté correcto. Se otorgará una parte proporcional en caso de tener desarrollo detallado, pero el resultado no sea el esperado.

Anota en la hoja tu nombre completo, así como una identificación de cada ejercicio.

\begin{enumerate}
\item Un cubo de aluminio presenta \SI{2}{\centi\meter} de longitud en uno de sus lados y tiene una masa de \SI{21.2}{\gram}.

Responde las siguientes preguntas:
\begin{enumerate}[label=\alph*)]
\item ¿Cuál es su densidad?
\item ¿Cuál será la masa de \SI{5.5}{\cubic\centi\meter}?
\end{enumerate}
\item Un objeto Y tiene una masa de \SI{150}{\gram} y una densidad de \SI{2}{\gram\per\cubic\centi\meter}, un objeto Z tiene una masa de \SI{750}{\gram} y una densidad de \SI{10}{\gram\per\cubic\centi\meter}.
\begin{enumerate}[label=\alph*)]
\item Si se introducen por separado los dos objetos en un recipiente con agua, determina cuál desplazará mayor volumen de agua.
\item ¿Es posible que el objeto Y y el objeto Z sean de la misma sustancia? Sí o no y por qué.
\end{enumerate}
\item Calcula la profundidad a la que se encuentra sumergido un submarino en el mar, cuando soporta una presión hidrostática de \SI{8d6}{\newton\per\square\meter}. La densidad del agua de mar es de \SI{1020}{\kilo\gram\per\cubic\meter}.
\item Calcula el diámetro que debe tener el émbolo mayor de una prensa hidráulica para obtener una fuerza cuya magnitud es de \SI{2000}{\newton}, cuando el émbolo menor tiene un diámetro de \SI{10}{\centi\meter} y se aplica una fuerza cuya magnitud es de \SI{100}{\newton}.
\item Un cubo de acero de \SI{18}{\centi\meter} de arista se sumerge totalmente en agua. Si la magnitud de su peso es de \SI{480}{\newton}, ¿qué magnitud de empuje recibe?
\end{enumerate}


\end{document}