\documentclass[14pt]{extarticle}
\usepackage[utf8]{inputenc}
\usepackage[T1]{fontenc}
\usepackage[spanish,es-lcroman]{babel}
\usepackage{amsmath}
\usepackage{amsthm}
\usepackage{physics}
\usepackage{tikz}
\usepackage{float}
\usepackage[autostyle,spanish=mexican]{csquotes}
\usepackage[per-mode=symbol]{siunitx}
\usepackage{gensymb}
\usepackage{multicol}
\usepackage{enumitem}
\usepackage[left=2.00cm, right=2.00cm, top=2.00cm, 
     bottom=2.00cm]{geometry}
\usepackage{Estilos/ColoresLatex}
\usepackage{makecell}

\newcommand{\textocolor}[2]{\textbf{\textcolor{#1}{#2}}}
\DeclareSIUnit[number-unit-product = {\,}]\cal{cal}

%\renewcommand{\questionlabel}{\thequestion)}
\decimalpoint
\sisetup{bracket-numbers = false}

\title{\vspace*{-2cm} Ejercicios Campo Eléctrico- Física 2\vspace{-5ex}}
\date{\today}

\begin{document}
\maketitle

\section{Ejercicios a cuenta}

Resuelve de manera detallada los siguientes ejercicios.

\vspace*{0.5cm}
\textbf{LA SOLUCIÓN DEBE DE ESTAR HECHA A MANO}.

\vspace*{0.5cm}
Esta actividad te otorgará hasta  $7$ puntos a la Evaluación Continua.

La entrega se hará vía Teams en asignación, teniendo como plazo el día domingo 6 de agosto a las 8 pm.

Cada ejercicio vale $1$ punto, siempre y cuando esté correcto. Se otorgará una parte proporcional en caso de tener desarrollo detallado, pero el resultado no sea el esperado.

Anota en la hoja tu nombre completo, así como el enunciado completo de cada ejercicio.

\begin{enumerate}
\item Una carga de prueba de \SI{3d-7}{\coulomb} recibe una fuerza horizontal hacia la derecha de \SI{2d-4}{\newton}. ¿Cuál es la magnitud de la intensidad del campo eléctrico en el punto donde está colocada la carga de prueba?
\item Una carga de prueba de \SI{2}{\micro\coulomb} se sitúa en un punto en el que la intensidad del campo eléctrico tiene una magnitud de \SI{5d2}{\newton\per\coulomb}. ¿Cuál es la magnitud de la fuerza que actúa sobre ella?
\item Calcula la magnitud de la intensidad del campo eléctrico a una distancia de \SI{50}{\centi\meter} de una carga de \SI{4}{\micro\coulomb}.
\item La intensidad del campo eléctrico producido por una carga de \SI{3}{\micro\coulomb} en un punto determinado tiene una magnitud de \SI{6d6}{\newton\per\coulomb}. ¿A qué distancia del punto considerado se encuentra la carga?
\item Una esfera metálica, cuyo diámetro es de \SI{20}{\centi\meter}, está electrizada con una carga de \SI{8}{\micro\coulomb} distribuida uniformemente en su superficie. ¿Cuál es la magnitud de la intensidad del campo eléctrico a \SI{8}{\centi\meter} de la superficie de la esfera?
\item Determina la carga transportada desde un punto a otro al realizarse un trabajo de \SI{3d-3}{\joule}, si la diferencia de potencial es \SI{4d2}{\volt}.
\item Para transportar una carga de \SI{9}{\micro\coulomb} desde el suelo hasta la superficie de una esfera cargada se realiza un trabajo de \SI{7d-5}{\joule}. ¿Cuál es el potencial eléctrico de la esfera?
\end{enumerate}


\end{document}